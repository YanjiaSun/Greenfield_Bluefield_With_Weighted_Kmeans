%%  Last update 7/03
%%%%%%%%%%%%%%%%%%%%%%%%%%%%%%%%%%%%%%%%%%%%%%%%%%%%%%%%%%%%%%
%%  This macro contains definitions of new \LaTeX commands  %%
%%  that are frequently used by Barry L. Nelson.  This is   %%
%%  a new release of the 'latexmacro.tex' file that I have  %%
%%  distributed for some time.  This release                %%
%%  is designed to be compatible with previous releases,    %%
%%  except for eliminating the \smallheads command.         %%
%%                                                          %%
%%  This releases fixes a few bugs (please report any you   %%
%%  find to nelsonb@random.eng.ohio-state.edu), includes    %%
%%  more documentation, and completes the definitions of    %%
%%  bold-face mathematics characters.                       %%
%%                                                          %%
%%  release 2.00   12/30/91                                 %%
%%                                                          %%
%%%%%%%%%%%%%%%%%%%%%%%%%%%%%%%%%%%%%%%%%%%%%%%%%%%%%%%%%%%%%%
%%
%%
%% Temporary fix for the paper size problem
\special{papersize=8.5in,11in}
%%
%% USING THE MACRO:
%% If you name this file latexmacro.tex, then use the following
%% statement just after the \documentstyle{} command:
%%
%%    \input [path] latexmacro
%%
%% Packages to add when using MikTex
%\usepackage{amsthm}
%%
%% LINE SPACING:
%% Define single, one and a half, and double space commands; these
%% may be used anywhere after \begin{document}
%%
\def\double{\par\baselineskip=24pt}
\def\single{\par\baselineskip=12pt}
\def\oneandahalf{\par\baselineskip=18pt}
%%
%% TEXT DIMENSIONS
%% Define text widths \standard, \proceedings and \myletter.
%% \proceedings prints a single, narrow column near the edge of the page.
%% Use these before \begin{document} command.
%%
%% Use \enumer for full width page when all text uses \enumerate (MP 9/94)
%%
\def\enumer{\oddsidemargin=-.2in
              \evensidemargin=-.2in
              \topmargin =-.5in
              \textheight=8.5in
              \textwidth=6.7in}
\def\standard{\oddsidemargin=0in
              \evensidemargin=0in
              \topmargin =-.5in
              \textheight=8.5in
              \textwidth=6.5in}
\def\proceedings{\oddsidemargin=-.5in
                 \evensidemargin=-.5in
                 \textwidth=4.56in
                 \parindent=0.3in
                 \topmargin=-1in
                 \textheight=10in
                 \pagestyle{empty}}
\def\myletter{\oddsidemargin=.2in
              \evensidemargin=.2in
              \textwidth=6in
              \textheight=8in
              \topmargin=-.5in
              \baselineskip=16pt}

%                 \oddsidemargin=.3in
%                 \evensidemargin=.3in
%                 \textwidth=5.5in}

% For documents where everything is in enumerate environment  (MP)
\def\enumber{\oddsidemargin=-.2in
             \evensidemargin=-.2in
             \topmargin =-.5in
             \textheight=8.5in
             \textwidth=6.7in
             \parindent=0in}
%
% For documents where item is used to number problems  (MP)
% Use \itemz
%
\def\eitem{\oddsidemargin=.05in
             \evensidemargin=.05in
             \topmargin =-.5in
             \textheight=8.5in
             \textwidth=6.5in
             \parindent=0in}
%
\def\itemz#1{\par\noindent\hangindent=0mm \hskip -1.1mm \llap{#1\enspace}}
%%
%% Define command to pull in letterhead file.
%% This command is specific to VAX/VMS and Arbortext dvips program.
%% Insert your own path.
%%
\def\letterhead{\special{ps: plotfile [is.posner.com]letterhead.tex local}}
%%
%%
%  Defines command to use OSU Letterhead paper (bottom tray of HP LaserJet)
%
\def\osuletterhead{\special{ps::[local]
    statusdict begin
    0 setpapertray
    end}}
%%
%% LISTS:
%% Define \plainlist environment
%%
\newenvironment{plainlist}
{\begin{list}%
{}{\setlength{\leftmargin}{\rightmargin}\setlength{\labelwidth}{0in}}}%
{\end{list}}
%%
%%
%% Define \indentlist environment (like plainlist but indents) MP
%%
\newenvironment{indentlist}
{\begin{list}%
{}{\setlength{\leftmargin}{7.1mm}\setlength{\labelwidth}{1in}}}%
{\end{list}}
%%
%%
%% Define \numerate environment (like enumerate but no indents) MP
%%
\newcounter{bean}
\newenvironment{numerate}
{\begin{list}%
{\arabic{bean}.}{\usecounter{bean}\setlength{\leftmargin}{7.1mm}%
\setlength{\rightmargin}{0pt}}}%
{\end{list}}
%%
%% Define \hangref environment for hanging indentation. I MODIFY FOR MORE SPACE
%%                                                      BETWEEN LINES
%%
\newenvironment{hangref}{\begin{list}{}{\setlength{\leftmargin}{+\parindent}
\setlength{\itemindent}{-\parindent}}}{\end{list}}
%%
%%
%%
%%
%% MATHEMATICS:
%%
%%
% Equationarray.doc
%
% Combine the capabilities of the eqnarray environment and the array
% environment.
%
% Problem: The eqnarray environment is too restrictive in that it is
% only prepared to accept three part equations.  Some applications
% require more sophisticated mathematics, so three parts aren't
% enough.  Simultaneously, we would like to have our equations
% numbered.  If the equations aren't to be numbered, then the array
% environment suffices.  If the equations aren't to be aligned with
% each other, then the equation environment is preferred.
%
% Usage:
%
%   \begin{equationarray}{rrcl}
%   % Equations go here, in the same format as you would use
%   % for arrays.
%   \end{equationarray}
%
% Written by:
%   Tony Li, University of Southern California, <tli@sargas.usc.edu>
%   Started 6/15/88.
%   Yes, I know this is a kludge.  Suggestions accepted.
%
% Notes: This was a style file.  The \makeatletter command enables it to
%        be put in an input file.
%
\makeatletter

\def\equationarray{\stepcounter{equation} \let\@currentlabel=\theequation
 \global\@eqnswtrue \global\@eqcnt\z@ \global\@eqargcnt \z@ \let\\=\@equationcr
 \let\@acol\@arrayacol \let\@classz\@eqnclassz \def\@halignto{to\displaywidth}
 \@equationarray}

\def\@equationarray#1{\setbox\@arstrutbox=\hbox{\vrule
     height\arraystretch \ht\strutbox
     depth\arraystretch \dp\strutbox
     width\z@}
     \@mkpream{#1}
     \edef\@preamble{\halign \noexpand\@halignto
      \bgroup \tabskip\z@ \@arstrut \@preamble
      \hfill\tabskip\@centering &\llap{\@sharp}\tabskip\z@\cr}%
     \let\@startpbox\@@startpbox \let\@endpbox\@@endpbox
     \bgroup \let\par\relax
     \let\@sharp## \let\protect\relax \lineskip\z@ \baselineskip\z@
     \tabskip\@centering $$ % $$ BRACE MATCHING HACK
     \@preamble
}

\def\@eqnclassz{\ifcase \@lastchclass \@acolampacol \or \@ampacol \or
   \or \or \@addamp \or
   \@acolampacol \or \@firstampfalse \@acol \fi
   \global\advance\@eqargcnt\@ne
   \edef\@preamble{\@preamble
     \ifcase \@chnum
     \hfil$\relax\displaystyle\tabskip\z@\@sharp$\hfil \or
     $\relax\displaystyle\tabskip\z@\@sharp$\hfil \or
     \hfil$\relax\displaystyle\tabskip\z@\@sharp$\fi
     \global\advance\@eqcnt\@ne}}

\def\endequationarray{\@equationcr
\egroup\global\advance\c@equation\m@ne % $$ BRACE MATCHING HACK
$$\tabskip\@centering\egroup\global\@ignoretrue}

\def\@equationcr{${\ifnum0=`}\fi\@ifstar{\@xequationcr}{\@xequationcr}}
\def\@xequationcr{
    \@ifnextchar[{\@argequationcr}{\ifnum0=`{\fi}${}\@zequationcr
     \global\@eqnswtrue\global\@eqcnt\z@\cr}}

\def\@argequationcr[#1]{\ifnum0=`{\fi}${}\ifdim #1>\z@
   \@xargequationcr{#1}\else
   \@yargequationcr{#1}\fi}

\def\@xargequationcr#1{
   \@tempdima #1\advance\@tempdima \dp \@arstrutbox
   \vrule \@height\z@ \@depth\@tempdima \@width\z@ \@zequationcr
     \global\@eqnswtrue\global\@eqcnt\z@ \cr}

\def\@yargequationcr#1{ \@zequationcr
     \global\@eqnswtrue\global\@eqcnt\z@ \cr\noalign{\vskip #1}}

\def\@zequationcr{\@wargequationcr\if@eqnsw\@eqnnum
    \stepcounter{equation}\global\@eqcnt\z@\fi}

\def\@wargequationcr{\ifnum\@eqcnt<\@eqargcnt
    \@amper \@wargequationcr\fi}

\def\@amper{&}

\newcount\@eqargcnt

\makeatother
%%
%%
%%Special math symbols
%%
%% operators
\def\E{{\rm E}}
\def\Var{{\rm Var}}
\def\Sd{{\rm Sd}}
\def\Cov{{\rm Cov}}
\def\Corr{{\rm Corr}}
\def\Bias{{\rm Bias}}
\def\MSE{{\rm MSE}}
\def\CP{{\rm CP}}
\def\trace{{\rm trace}}
%%
%% bold face roman
\def\bA{{\bf A}}
\def\ba{{\bf a}}
\def\bC{{\bf C}}
\def\bc{{\bf c}}
\def\bD{{\bf D}}
\def\bd{{\bf d}}
\def\bE{{\bf E}}
\def\be{{\bf e}}
\def\bF{{\bf F}}
\def\boldf{{\bf f}}
\def\bG{{\bf G}}
\def\bg{{\bf g}}
\def\bH{{\bf H}}
\def\bh{{\bf h}}
\def\bi{{\bf i}}
\def\bI{{\bf I}}
\def\bJ{{\bf J}}
\def\bj{{\bf j}}
\def\bK{{\bf K}}
\def\bk{{\bf k}}
\def\bL{{\bf L}}
\def\bl{{\bf l}}
\def\bM{{\bf M}}
\def\bm{{\bf m}}
\def\bN{{\bf N}}
\def\bn{{\bf n}}
\def\bO{{\bf O}}
\def\bo{{\bf o}}
\def\bP{{\bf P}}
\def\bp{{\bf P}}
\def\bQ{{\bf Q}}
\def\bq{{\bf q}}
\def\bR{{\bf R}}
\def\br{{\bf r}}
\def\bS{{\bf S}}
\def\bs{{\bf s}}
\def\bU{{\bf U}}
\def\bu{{\bf u}}
\def\bV{{\bf V}}
\def\bv{{\bf v}}
\def\bW{{\bf W}}
\def\bw{{\bf w}}
\def\bX{{\bf X}}
\def\bx{{\bf x}}
\def\bY{{\bf Y}}
\def\by{{\bf y}}
\def\bZ{{\bf Z}}
\def\bz{{\bf z}}
%%
%% bold face Greek (incomplete)
\def\bbeta{\mbox{\boldmath $\beta$}}
\def\bepsilon{\mbox{\boldmath $\epsilon$}}
\def\bvarepsilon{\mbox{\boldmath $\varepsilon$}}
\def\bDelta{{\bf \Delta}}
\def\bdelta{\mbox{\boldmath $\delta$}}
\def\boldeta{\mbox{\boldmath $\eta$}}
\def\bGamma{{\bf \Gamma}}
\def\bgamma{\mbox{\boldmath $\gamma$}}
\def\bLambda{{\bf \Lambda}}
\def\blambda{\mbox{\boldmath $\lambda$}}
\def\bmu{\mbox{\boldmath $\mu$}}
\def\bOmega{{\bf \Omega}}
\def\bphi{\mbox{\boldmath $\phi$}}
\def\bpsi{\mbox{\boldmath $\psi$}}
\def\bSigma{{\bf \Sigma}}
\def\bsigma{\mbox{\boldmath $\sigma$}}
\def\bTheta{{\bf \Theta}}
\def\btheta{\mbox{\boldmath $\theta$}}
\def\bXi{{\bf \Xi}}
\def\bxi{\mbox{\boldmath $\xi$}}
%%
%% misc. Greek with hats
\def\betahat{\hat \beta}
\def\bbetahat{\hat{\mbox{\boldmath $\beta$}}}
\def\bdeltahat{\hat{\mbox{\boldmath $\delta$}}}
\def\epsilonhat{\hat{\epsilon}}
\def\bepsilonhat{\hat{\mbox{\boldmath $\epsilon$}}}
\def\bgammahat{\hat{\mbox{\boldmath $\gamma$}}}
\def\bmuhat{\hat{\mbox{\boldmath $\mu$}}}
\def\thetahat{\hat \theta}
%%
%% bold face numbers
\def\b1{{\bf 1}}
\def\bzero{{\bf 0}}
%%
%% Gives period at start of proof\newtheorem{lemma}{Lemma}
\newcommand{\pf}{\vspace{-6pt} {\noindent \bf Proof. }}
%%

%% \blot gives a square at end of formulas and proofs
%% (thanks to Marc Posner)
\def\blot{\quad \mbox{$\vcenter{\vbox{\hrule height.4pt
    \hbox{\vrule width.4pt height.9ex \kern.9ex \vrule width.4pt}
    \hrule height.4pt}}$}}

\def\Box{\quad \mbox{$\vcenter{\vbox{\hrule height.4pt
    \hbox{\vrule width.4pt height.9ex \kern.9ex \vrule width.4pt}
    \hrule height.4pt}}$}}

\def\qed{\quad \mbox{$\vcenter{\vbox{\hrule height.4pt
    \hbox{\vrule width.4pt height.9ex \kern.9ex \vrule width.4pt}
    \hrule height.4pt}}$}}
%%

% making set union symbol with limits look elegant
\def\union{\mathop{\cup}\limits}

%%
%
% Smaller headings for sections in documents
\def\smallheads{
\renewcommand{\Large}{\large}   % reduces to next smaller size
\renewcommand{\LARGE}{\Large}}  % reduces to next smaller sizes
%
%
% gives space before and after |
\def\st{\: | \:}
%
\def\argmin{{\rm argmin}}
\def\argmax{{\rm argmax}}
%
%
%  New list macros and modified old list macros. - JJC 09/93
%
% gives the vector 1, 2, ..., n where the n is controllable
\def\irow#1{1, 2, \ldots, #1}
%
% gives the vector 0, 1, ..., n where n is controllable
\def\irowz#1{0, 1, \ldots, #1}
%
% gives the vector i = 1, 2, ..., n where i and n are controllable
\def\irowl#1#2{#1 = 1, 2, \ldots, #2}
%
% gives the vector i = 0, 1, ..., n where i and n are controllable
\def\irowlz#1#2{#1 = 0, 1, \ldots, #2}
%
% gives the vector x_1, x_2, ..., x_n where the x and n are controllable
\def\row#1#2{#1_1, #1_2, \ldots, #1_{#2}}
%
% gives the vector x_0, x_1, ..., x_n where the x and n are controllable
\def\rowz#1#2{#1_0, #1_1, \ldots, #1_{#2}}
%
% gives the vector y = x_1, x_2, ..., x_n where the y, x and n are controllable
\def\rowl#1#2#3{#1 = #2_1, #2_2, \ldots, #2_{#3}}
%
% gives the vector y = x_0, x_1, ..., x_n where the y, x and n are controllable
\def\rowlz#1#2#3{#1 = #2_0, #2_1, \ldots, #2_{#3}}
%%%%%
%
% gives symbol for real line
\def\real{I\!\!R}
%
%gives symbol for set of integers
\def\integer{Z\!\!\!Z}
%
% \cents gives cent symbol
\def\cents{\mbox{\hbox{\rm\rlap/c \hspace{2pt}}}}
%
% gives shortcut for doing et al. in italics
\def\etal{{\em et al.\ }}
\def\etalc{{\em et al.}, }
%
% {\bfnoindent} gives boldface and prevents indentation- for Theorems,
% Proofs, etc.
\def\bfnoindent{\bf\noindent}
%
% Redefines \item to give more versatility. itema=itemitem, itemb=3 items
% where each item is 4mm
\def\itema#1{\par\noindent\hangindent=4mm \hskip 2.4mm \llap{#1\enspace}}
\def\itemb#1{\par\noindent\hangindent=8mm \hskip 6.4mm \llap{#1\enspace}}
\def\itemc#1{\par\noindent\hangindent=12mm \hskip 10.4mm \llap{#1\enspace}}
\def\itemd#1{\par\noindent\hangindent=16mm \hskip 14.4mm \llap{#1\enspace}}
\def\iteme#1{\par\noindent\hangindent=20mm \hskip 18.4mm \llap{#1\enspace}}
\def\itemf#1{\par\noindent\hangindent=24mm \hskip 22.4mm \llap{#1\enspace}}
\def\itemg#1{\par\noindent\hangindent=28mm \hskip 26.4mm \llap{#1\enspace}}
\def\itemh#1{\par\noindent\hangindent=32mm \hskip 30.4mm \llap{#1\enspace}}
%
% Creates hangindents for defs and slides
\newcommand{\hangr}{\hangindent=12pt}
\newcommand{\hangd}{\hangindent=24pt}
%
%
% THESE COMMANDS ARE NO LONGER NEEDED BUT MAY EXIST IN OLD FILES
%
% Item used for references with no numbering
\def\ritem{\par\noindent\hangindent=4mm \parskip=12pt}
%
% \bull gives a square at end of formula
\def\bull{\quad {$\vcenter{\vbox{\hrule height.4pt
             \hbox{\vrule width.4pt height.9ex \kern.9ex \vrule width.4pt}
             \hrule height.4pt}}$}}
%
% use in displayed equations :\ebull gives a square at end of formula
\def\ebull{\quad {\vcenter{\vbox{\hrule height.4pt
             \hbox{\vrule width.4pt height.9ex \kern.9ex \vrule width.4pt}
             \hrule height.4pt}}}}
% gives a more bold *
\def\bfast{{*\kern-.38em *}}
%
% gives a big *, even if subscript or superscript
\def\bast{{\textstyle *}}
